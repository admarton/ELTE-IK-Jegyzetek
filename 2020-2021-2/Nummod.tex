\documentclass{article}
\usepackage[margin=0.75in]{geometry}
\usepackage{amsmath}
\usepackage{amssymb}

\usepackage{hyperref}
\hypersetup{
    colorlinks=false, %set true if you want colored links
    linktoc=all,     %set to all if you want both sections and subsections linked
    linkcolor=blue,  %choose some color if you want links to stand out
}

\setlength{\parindent}{0em}
%\setlength{\parskip}{5em}
%\setlength{\parbox}{5em}
\def\thesection{\arabic{section}.}
\def\thesubsection{\arabic{section}.\alph{subsection}.}
\def\thesubsubsection{\arabic{section}.\alph{subsection}.\arabic{subsubsection}.}

\title{Nummod I. - Tételek}
\date{2021-05-27}
\author{Ambrus-Dobai Márton}

\begin{document}
    \pagenumbering{arabic}
    \maketitle

    \tableofcontents

    \section{Lebegőpontos számok és tulajdonságaik. A Horner-algoritmus.}
        \subsection{
        Ismertesse a lebegőpontos számábrázolás modelljét, és definiálja a gépi számokat.
        Nevezze meg és számítsa ki a számhalmaz nevezetes mennyiségeit (elemszám, $M\infty$, $\varepsilon_0$).
        Szemléltesse a halmaz elemeit számegyenesen. Adjon meg két példát a véges szám-ábrázolásból fakadó furcsaságokra.
        }

            \subsubsection{Ismertesse a lebegőpontos számábrázolás modelljét, és definiálja a gépi számokat.}

            \textbf{Definíció:} Normalizált lebegőpontos szám

            Legyen $m = \sum^{t}_{i=1} m_j\cdot2^{-i}$, ahol $ t \in \mathbb{N}, m_1 = 1, m_i \in \{0,1\}.$ ~\\
            Ekkor az $a = \pm m \cdot 2^k (k \in \mathbb{Z})$ alakú számot \underline{\textit{normalizált lebegőpontos számnak}} nevezzük. ~\\
            m: a szám \underline{mantisszája}, hossza t ~\\
            k: a szám \underline{karakterisztikája}, $k^- \leq k \leq k^+$ ~\\
            $101011 \rightsquigarrow 0.101011\cdot2^{-6}$ ~\\
            \textbf{Jelölés:} $a = \pm[m_1...m_k|k] = \pm0.m_1...m_k * 2^k$. ~\\
            
            \textbf{Jelölés:} $M = M(t, k^-, k^+)$ a gépi számok halmaza, adott $k^-,k^+ \in \mathbb{Z}$ és $t \in \mathbb{N}$ esetén. (Általában $k^-<0$ és $k^+>0$). \\
            \textbf{Definíció:} Gépi számok halmaza ~\\
            \[M(t, k^-, k^+) = \{ a = \pm 2^k\cdot\sum^{t}_{i=1} m_i\cdot 2^{-i} : k^- \leq k \leq k^+, m_i \in \{0,1\}, m_1=1  \}\bigcup\{0\}\]
            Gyakorlatban még hozzávesszük: $\infty, -\infty, NaN, ...$

            \subsubsection{Nevezze meg és számítsa ki a számhalmaz nevezetes mennyiségeit (elemszám, $M\infty$, $\varepsilon_0$).}

            \begin{itemize}
                \item $\frac{1}{2} \leq m \leq 1$
                \item M szimmetrikus a 0-ra.
                \item M legnagyobb eleme: \[M_\infty = [11...11|k^+] = 1.00...00\cdot2^{k^+}-0.00...01\cdot2^{k^+} = (1-2^{-t})\cdot2^{k^+}\]
                \item M legkisebb pozitív eleme: \[\varepsilon_0 = [10...00|k^-] = \frac{1}{2} \cdot 2^{k^-} = 2^{k^--1}\]
            \end{itemize}


        \subsection{Az input függvény fogalma, tétel az ábrázolt szám hibájáról, $\varepsilon_1$ mennyiség bevezetése és értelmezése.}
        
        
        \subsection{* A Horner algoritmus polinom és deriváltja helyettesítési értékeinek gyors számolására, és a kapcsolódó tétel igazolása.}


    \section{Becslés polinom gyökeinek elhelyezkedésére. A hibaszámítás alapjai.}
        \subsection{* Becslés polinom gyökeinek elhelyezkedésére és annak bizonyítása.}
        \subsection{Ismertesse az abszolút és relatív hiba, hibakorlát fogalmát. Mutassa be az alapműveletek hibakorlátaira vonatkozó állításokat, és igazolja a szorzásra vagy osztásra vonatkozó összefüggéseket. Ez alapján mely műveletek elvégzése veszélyes az abszolút és relatív hibára nézve és miért?}
        \subsection{Igazolja a függvényérték hibakorlátaira vonatkozó tételeket és definiálja függvény adott pontbeli kondíciószámát.}


    \section{A Gauss-elimináció és az LU-felbontás algoritmusa.}
        \subsection{Vázolja a Gauss-elimináció alapötletét LER megoldására, vezesse le az algoritmus képleteit. Mutassa be a Gauss-elimináció további alkalmazásait azonos mátrixú lineáris egyenletrendszerek megoldására, determináns kiszámítására és inverzmátrix meghatározására.}
        \subsection{Határozza meg az elimináció és a visszahelyettesítés műveletigényét.}
        \subsection{* Mutassa meg, hogy a GE lépései végrehajthatók speciális mátrix-szorzásokkal. Vezesse le a kapott mátrixok inverzére és szorzatára tanult állításokat. Végül ezeket felhasználva állítsa elő a kiinduló mátrix LU-felbontását.}
asd

    \section{A Gauss-elimináció és az LU-felbontás elemzése.}
        \subsection{Mutassa be a Gauss-elimináció algoritmusát. Adjon szükséges és elégséges feltételeket a GE elakadására illetve végrehajthatóságára. Ismertesse LER megoldását LU-felbontás segítségével. Miért előnyös ennek használata a GE-vel szemben?}
        \subsection{Ismertesse a részleges és teljes főelemkiválasztás módszereit. Mit mondhatunk az elakadásról részleges főelemkiválasztás alkalmazása esetén? Miért lehet érdemes teljes főelemkiválasztást használni?}
        \subsection{* Idézze fel az LU felbontás előállításának módszerét a Gauss-elimináció segítségével (bizonyítás nélkül). Adjon szükséges és elégséges feltételt a létezésre. Igazolja az LU-felbontás egyértelműségére vonatkozó tételt.}


    \section{Az LU-felbontás alkalmazása. A Schur-komplementer.}
        \subsection{Definiálja egy mátrix LU-felbontását. Adjon módszert L és U mátrixok elemenkénti meghatározására, vezesse le az elemekre vonatkozó képleteket. Térjen ki az elemek meghatározásának sorrendjére és a műveletigényre is.}
        \subsection{Definiálja a Schur-komplementert. Ismertesse a GE megmaradási tételeit (és a kapcsolódó fogalmakat), majd bizonyítsa a determinánsra és szimmetriára vonatkozó pontokat.}
        \subsection{* Igazolja a pozitív definitség megmaradására vonatkozó állítást.}
    
ads
    \section{A Cholesky-féle felbontás.}
        \subsection{Az LDU-felbontás fogalma, előállítása. Szimmetrikus mátrix felbontására vonatkozó tétel.}
        \subsection{* Definiálja a Cholesky-felbontást, igazolja a létezésére és egyértelműségére vonatkozó tételeket.}
        \subsection{Mutassa be az elemenkénti meghatározásra szolgáló (Cholesky-)algoritmust, vezesse le a képleteket és határozza meg a műveletigényt, tárigényt. Vesse össze az LDLT és a Cholesky felbontások alkalmazhatóságát.}


    \section{A QR-felbontás.}
        \subsection{Definiálja a QR-felbontást és vezesse le az előállítására alkalmas Gramm–Schmidt ortogonalizációs eljárást. Milyen feltétel garantálja, hogy az algoritmus nem akad el?}
        \subsection{Mutassa be az ortogonalizációs eljárás normálás nélküli változatát, és az utólagos normálás módját. Hogyan alkalmazható a QR-felbontás LER megoldására? Vesse össze az LU-felbontáson alapuló megoldással (műveletigény, alkalmazhatóság).}
        \subsection{* A QR-felbontás egyértelmúségére vonatkozó tétel.}


    \section{A Householder-transzformáció.}
        \subsection{Definiálja a Householder-transzformációt, ismertesse geometriai tartalmát, vezesse le elemi tulajdonságait. Mutassa be a transzformáció alkalmazásának módját vektorra illetve mátrixra (mindkét irányból), adja meg e számítások műveletigényeit.}
        \subsection{Határozza meg különböző, azonos (de nem 0) hosszúságú a, b vektorokhoz azt a H transzformációt, melyre $Ha = b$. Alkalmazza ezt az eredményt tetszőleges vektor $\sigma e_1$ alakra hozására, indokolja $\sigma$ értékének megválasztását.}
        \subsection{* Mutassa be a Householder-transzformáció alkalmazását lineáris egyenletrendszer megoldására, valamint QR-felbontás elkészítésére. Vesse össze a módszert a Gram–Schmidt eljárással műveletigény és numerikus stabilitás szempontjából.}

ads
    \section{Mátrixnormák és tulajdonságaik I.}
        \subsection{Definiálja a vektornormát, mátrixnormát és indukált normát, mutassa meg, hogy utóbbi mindig mátrixnorma. Adjon meg példákat is. Igazolja mátrix tetszőleges normája és a spektrálsugara közti egyenlőtlenséget.}
        \subsection{* Vezesse le a 2-es vektornorma által indukált mátrixnorma képletét.}
        \subsection{Vezesse le, mivel egyenlő normális mátrix 2-es normája. Mit mondhatunk ortogonális mátrix 2-es normájáról?}


    \section{Mátrixnormák és tulajdonságaik II.}
        \subsection{Definiálja a vektornormát, mátrixnormát és indukált normát, mutassa meg, hogy utóbbi mindig mátrixnorma. Írja fel a Frobenius mátrixnorma képletét (bizonyítás nélkül), igazolja, hogy nem indukált norma. Definiálja az illeszkedés fogalmát, igazolja, hogy indukált norma mindig illeszkedik a megfelelő vektornormához.}
        \subsection{* Igazolja a Frobenius-norma sajátértékekkel való kifejezésének képletét. Ezt felhasználva mutasson példát olyan vektornormára, melyhez a Frobenius-norma illeszkedik.}
        \subsection{Vezesse le az 1-es vektornorma által indukált mátrixnorma képletét.}

    
    \section{LER érzékenysége.}
        \subsection{Formalizálja LER jobboldalának illetve mátrixának perturbációját, ismertesse a megoldás megváltozásának mértékére tanult tételeket (bizonyítás nélkül). Definiálja a kondíciószámot és igazolja tulajdonságait.}
        \subsection{Vizsgálja LER megoldásának érzékenységét szorzatfelbontások (LU, QR) alkalmazása esetén. Bizonyítsa a LER jobboldalának megváltozására vonatkozó tételt.}
        \subsection{* Bizonyítsa a mátrix megváltozására vonatkozó tételt és a felhasznált lemmát.}

asd
    \section{Iterációs módszerek konvergenciája.}
        \subsection{* Kontrakció fogalma $\mathbb{R}^n$-en, a Banach-féle fixponttétel ismertetése és bizonyítása.}
        \subsection{Vázolja LER iterációs módszerrel történő megoldásának alapötletét, vesse össze a direkt módszerekkel. Vezessen le elégséges feltételt a konvergenciára.}
        \subsection{Igazolja a konvergencia szükséges és elégségséges feltételét.}


    \section{A Jacobi-iteráció.}
        \subsection{Vezesse le a Jacobi-iteráció mátrixos és koordinátás alakját. Ismertesse a csillapított változat alapötletét, határozza meg vektoros és koordinátás képleteit.}
        \subsection{Írja át az iterációt a reziduumvektoros alakra, térjen ki annak szerepére. Adjon elégséges feltételt a Jacobi-iteráció konvergenciájára.}
        \subsection{* Igazolja a csillapított Jacobi-iteráció konvergenciatételét.}


    \section{A Gauss–Seidel-iteráció.}
        \subsection{Vezesse le a Gauss–Seidel-iteráció vektoros és koordinátás alakját. Ismertesse a relaxált változat alapötletét, határozza meg vektoros és koordinátás képleteit.}
        \subsection{* Írja át az iterációt a reziduumvektoros alakra, térjen ki annak szerepére. Bizonyítsa a relaxációs módszer konvergenciájának szükséges feltételét.}
        \subsection{Vesse össze a Jacobi és Gauss-Seidel típusú iterációkat. Ismertesse a speciális mátrixosztályok eseteire tanult tételeket (bizonyítás nélkül), értelmezze az eredményeket.}

asd
    \section{A Richardson-típusú iterációk. Kerekítési hibák az iterációkban.}
        \subsection{Vezesse le a Richardson-típusú iterációk képletét. Írja fel a reziduumvektoros alakot és ismertesse annak jelentőségét. Fogalmazza meg a tanult konvergenciatételt (bizonyítás nélkül).}
        \subsection{* Igazolja a konvergenciatételt.}
        \subsection{Vezesse le a kerekítési hibák hosszútávú hatását egy (általános) iterációs módszer alkalmazásakor.}


    \section{A részleges LU-felbontás és az ILU algoritmus.}
        \subsection{Definiálja a részleges LU-felbontást és vezesse le az ILU algoritmust. Írja át reziduumvektoros alakra is. Adjon nevezetes példákat az algoritmus speciális eseteiként.}
        \subsection{Vázolja a részleges LU-felbontás előállításának algoritmusát. Adjon elégséges feltételt a felbontás létezésére és egyértelműségére.}
        \subsection{* A felbontást előállító algoritmus helyességét megalapozó tétel bizonyítása.}


    \section{Nemlineáris egyenletek megoldása I.}
        \subsection{Ismertesse a nemlineáris egyenletek megoldásának feladatát. Mutassa be a megoldás létezését biztosító állításokat: írja fel a Bolzano-tételt (bizonyítás nélkül), és segítségével igazolja a Brouwer-féle fixpont-tételt.}
        \subsection{Kontrakció fogalma [a; b] intervallumon és a Banach-féle fixponttétel (bizonyítás nélkül). Igazolja az elégséges feltételt a kontrakcióra.}
        \subsection{* Ismertesse a konvergenciarend fogalmát és elemi tulajdonságait. Igazolja a fixpontiterációk magasabb rendű konvergenciájáról szóló tételt.}

asd
    \section{Nemlineáris egyenletek megoldása II.}
        \subsection{Vázolja az intervallumfelezés algoritmusát és mutasson hozzá hibabecslést. Ismertesse a húrmódszer alapötletét, szemléltesse működését és vezesse le az algoritmusát.}
        \subsection{Ismertesse a Newton-módszer alapötletét, szemléltesse működését és vezesse le a képletét. Mutassa be a többváltozós esetet is. Milyen tételt ismer a módszer monoton konvergenciájáról (bizonyítás nélkül)?}
        \subsection{* Igazolja a Newton-módszer monoton konvergenciájáról szóló állítást.}


    \section{Nemlineáris egyenletek megoldása III.}
        \subsection{Ismertesse a Newton-módszer alapötletét, szemléltesse működését és vezesse le a képletét. Mutassa be a többváltozós esetet is. Milyen tételt ismer a módszer lokális konvergenciájáról (bizonyítás nélkül)?}
        \subsection{* Igazolja a Newton-módszer lokális konvergenciájáról szóló állítást.}
        \subsection{Ismertesse a szelőmódszer alapötletét, szemléltesse működését és vezesse le az algoritmusát. Adjon konvergenciatételt (bizonyítás nélkül). Vesse össze az eredményeket a Newton-módszerről tanultakkal.}
    
        asdasd

\end{document}