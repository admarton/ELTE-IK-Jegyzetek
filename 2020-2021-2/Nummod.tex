\documentclass{article}
\usepackage[margin=0.75in]{geometry}

\setlength{\parindent}{4em}
\setlength{\parskip}{1em}

\title{Nummod I. - Tételek}
\date{2021-05-27}
\author{Ambrus-Dobai Márton}

\iffalse
# pdflatex

## 1. Lebegőpontos számok és tulajdonságaik. A Horner-algoritmus.
### 1. a)
Ismertesse a lebegőpontos számábrázolás modelljét, és definiálja a gépi számokat.
- **Def.:** Normalizált lebegőpontos szám
    - Legyen m = ∑ᵗⱼ₌₁ mⱼ * 2⁻ʲ, ahol t ∈ ℕ, m₁ = 1, mᵢ∈{0,1}.
    - Ekkor az a = ±m*2ᵏ (k∈ℤ) alakú számot _normalizált lebegőpontos számnak_ nevezzük.
    - m: a szám mantisszája, hossza t
    - k: a szám karakterisztikája, k⁻ ≤ k ≤ k⁺
    - 101011 ⇝ 0.101011*2⁻⁶
- **Jel.:** a = ±[m₁...mₜ|k] = ±0.m₁...mₜ * 2ᵏ.
- **Def.:** Gépi számok halmaza
    - 

Nevezze meg és számítsa ki a számhalmaz nevezetes mennyiségeit (elemszám, M∞, ε0).
\fi

\begin{document}
    \pagenumbering{arabic}
    \maketitle
    \section{Lebegőpontos számok és tulajdonságaik. A Horner-algoritmus.}

    \subsection{
    Ismertesse a lebegőpontos számábrázolás modelljét, és definiálja a gépi számokat.
    Nevezze meg és számítsa ki a számhalmaz nevezetes mennyiségeit (elemszám, $M\infty$, $\varepsilon_0$).
    Szemléltesse a halmaz elemeit számegyenesen. Adjon meg két példát a véges szám-ábrázolásból fakadó furcsaságokra.
    }

    \subsubsection{Ismertesse a lebegőpontos számábrázolás modelljét, és definiálja a gépi számokat.}
    \paragraph{Definíció: Normalizált lebegőpontos szám}
    
    Legyen $m = \sum^{t}_{i=1} m_j\cdot2^{-i}$, ahol $ t \in \mathbb{N}, m_1 = 1, m_i \in \{0,1\}$. 


    \par
\end{document}

\iffalse

b) Az input függvény fogalma, tétel az ábrázolt szám hibájáról, ε1 mennyiség bevezetése és értelmezése.
c) * A Horner algoritmus polinom és deriváltja helyettesítési értékeinek gyors számolására, és a kapcsolódó tétel igazolása.
2. Becslés polinom gyökeinek elhelyezkedésére. A hibaszámítás alapjai.
a) * Becslés polinom gyökeinek elhelyezkedésére és annak bizonyítása.
b) Ismertesse az abszolút és relatív hiba, hibakorlát fogalmát. Mutassa be az alapműveletek hibakorlátaira vonatkozó állításokat, és igazolja a szorzásra vagy osztásra
vonatkozó összefüggéseket. Ez alapján mely műveletek elvégzése veszélyes az abszolút és relatív hibára nézve és miért?
c) Igazolja a függvényérték hibakorlátaira vonatkozó tételeket és definiálja függvény
adott pontbeli kondíciószámát.
3. A Gauss-elimináció és az LU-felbontás algoritmusa.
a) Vázolja a Gauss-elimináció alapötletét LER megoldására, vezesse le az algoritmus
képleteit. Mutassa be a Gauss-elimináció további alkalmazásait azonos mátrixú
lineáris egyenletrendszerek megoldására, determináns kiszámítására és inverzmátrix
meghatározására.
b) Határozza meg az elimináció és a visszahelyettesítés műveletigényét.
c) * Mutassa meg, hogy a GE lépései végrehajthatók speciális mátrix-szorzásokkal.
Vezesse le a kapott mátrixok inverzére és szorzatára tanult állításokat. Végül ezeket
felhasználva állítsa elő a kiinduló mátrix LU-felbontását.
4. A Gauss-elimináció és az LU-felbontás elemzése.
a) Mutassa be a Gauss-elimináció algoritmusát. Adjon szükséges és elégséges feltételeket a GE elakadására illetve végrehajthatóságára. Ismertesse LER megoldását
LU-felbontás segítségével. Miért előnyös ennek használata a GE-vel szemben?
b) Ismertesse a részleges és teljes főelemkiválasztás módszereit. Mit mondhatunk az
elakadásról részleges főelemkiválasztás alkalmazása esetén? Miért lehet érdemes
teljes főelemkiválasztást használni?
c) * Idézze fel az LU felbontás előállításának módszerét a Gauss-elimináció segítségével
(bizonyítás nélkül). Adjon szükséges és elégséges feltételt a létezésre. Igazolja az
LU-felbontás egyértelműségére vonatkozó tételt.
5. Az LU-felbontás alkalmazása. A Schur-komplementer.
a) Definiálja egy mátrix LU-felbontását. Adjon módszert L és U mátrixok elemenkénti
meghatározására, vezesse le az elemekre vonatkozó képleteket. Térjen ki az elemek
meghatározásának sorrendjére és a műveletigényre is.
b) Definiálja a Schur-komplementert. Ismertesse a GE megmaradási tételeit (és a kapcsolódó fogalmakat), majd bizonyítsa a determinánsra és szimmetriára vonatkozó
pontokat.
c) * Igazolja a pozitív definitség megmaradására vonatkozó állítást.

6. A Cholesky-féle felbontás.
a) Az LDU-felbontás fogalma, előállítása. Szimmetrikus mátrix felbontására vonatkozó tétel.
b) * Definiálja a Cholesky-felbontást, igazolja a létezésére és egyértelműségére vonatkozó tételeket.
c) Mutassa be az elemenkénti meghatározásra szolgáló (Cholesky-)algoritmust, vezesse
le a képleteket és határozza meg a műveletigényt, tárigényt. Vesse össze az LDLT
és a Cholesky felbontások alkalmazhatóságát.
7. A QR-felbontás.
a) Definiálja a QR-felbontást és vezesse le az előállítására alkalmas Gramm–Schmidt
ortogonalizációs eljárást. Milyen feltétel garantálja, hogy az algoritmus nem akad
el?
b) Mutassa be az ortogonalizációs eljárás normálás nélküli változatát, és az utólagos
normálás módját. Hogyan alkalmazható a QR-felbontás LER megoldására? Vesse
össze az LU-felbontáson alapuló megoldással (műveletigény, alkalmazhatóság).
c) * A QR-felbontás egyértelmúségére vonatkozó tétel.
8. A Householder-transzformáció.
a) Definiálja a Householder-transzformációt, ismertesse geometriai tartalmát, vezesse
le elemi tulajdonságait. Mutassa be a transzformáció alkalmazásának módját vektorra illetve mátrixra (mindkét irányból), adja meg e számítások műveletigényeit.
b) Határozza meg különböző, azonos (de nem 0) hosszúságú a, b vektorokhoz azt a H
transzformációt, melyre Ha = b. Alkalmazza ezt az eredményt tetszőleges vektor
σe1 alakra hozására, indokolja σ értékének megválasztását.
c) * Mutassa be a Householder-transzformáció alkalmazását lineáris egyenletrendszer
megoldására, valamint QR-felbontás elkészítésére. Vesse össze a módszert a Gram–
Schmidt eljárással műveletigény és numerikus stabilitás szempontjából.
9. Mátrixnormák és tulajdonságaik I.
a) Definiálja a vektornormát, mátrixnormát és indukált normát, mutassa meg, hogy
utóbbi mindig mátrixnorma. Adjon meg példákat is. Igazolja mátrix tetszőleges
normája és a spektrálsugara közti egyenlőtlenséget.
b) * Vezesse le a 2-es vektornorma által indukált mátrixnorma képletét.
c) Vezesse le, mivel egyenlő normális mátrix 2-es normája. Mit mondhatunk ortogonális mátrix 2-es normájáról?
10. Mátrixnormák és tulajdonságaik II.
a) Definiálja a vektornormát, mátrixnormát és indukált normát, mutassa meg, hogy
utóbbi mindig mátrixnorma. Írja fel a Frobenius mátrixnorma képletét (bizonyítás
nélkül), igazolja, hogy nem indukált norma. Definiálja az illeszkedés fogalmát,
igazolja, hogy indukált norma mindig illeszkedik a megfelelő vektornormához.
b) * Igazolja a Frobenius-norma sajátértékekkel való kifejezésének képletét. Ezt felhasználva mutasson példát olyan vektornormára, melyhez a Frobenius-norma illeszkedik.
c) Vezesse le az 1-es vektornorma által indukált mátrixnorma képletét.
11. LER érzékenysége.
a) Formalizálja LER jobboldalának illetve mátrixának perturbációját, ismertesse a
megoldás megváltozásának mértékére tanult tételeket (bizonyítás nélkül). Definiálja
a kondíciószámot és igazolja tulajdonságait.
b) Vizsgálja LER megoldásának érzékenységét szorzatfelbontások (LU, QR) alkalmazása esetén. Bizonyítsa a LER jobboldalának megváltozására vonatkozó tételt.
c) * Bizonyítsa a mátrix megváltozására vonatkozó tételt és a felhasznált lemmát.
12. Iterációs módszerek konvergenciája.
a) * Kontrakció fogalma R
n
-en, a Banach-féle fixponttétel ismertetése és bizonyítása.
b) Vázolja LER iterációs módszerrel történő megoldásának alapötletét, vesse össze a
direkt módszerekkel. Vezessen le elégséges feltételt a konvergenciára.
c) Igazolja a konvergencia szükséges és elégségséges feltételét.
13. A Jacobi-iteráció.
a) Vezesse le a Jacobi-iteráció mátrixos és koordinátás alakját. Ismertesse a csillapított
változat alapötletét, határozza meg vektoros és koordinátás képleteit.
b) Írja át az iterációt a reziduumvektoros alakra, térjen ki annak szerepére. Adjon
elégséges feltételt a Jacobi-iteráció konvergenciájára.
c) * Igazolja a csillapított Jacobi-iteráció konvergenciatételét.
14. A Gauss–Seidel-iteráció.
a) Vezesse le a Gauss–Seidel-iteráció vektoros és koordinátás alakját. Ismertesse a
relaxált változat alapötletét, határozza meg vektoros és koordinátás képleteit.
b) *
Írja át az iterációt a reziduumvektoros alakra, térjen ki annak szerepére. Bizonyítsa a relaxációs módszer konvergenciájának szükséges feltételét.
c) Vesse össze a Jacobi és Gauss-Seidel típusú iterációkat. Ismertesse a speciális
mátrixosztályok eseteire tanult tételeket (bizonyítás nélkül), értelmezze az eredményeket.
15. A Richardson-típusú iterációk. Kerekítési hibák az iterációkban.
a) Vezesse le a Richardson-típusú iterációk képletét. Írja fel a reziduumvektoros alakot és ismertesse annak jelentőségét. Fogalmazza meg a tanult konvergenciatételt
(bizonyítás nélkül).
b) * Igazolja a konvergenciatételt.
c) Vezesse le a kerekítési hibák hosszútávú hatását egy (általános) iterációs módszer
alkalmazásakor.
16. A részleges LU-felbontás és az ILU algoritmus.
a) Definiálja a részleges LU-felbontást és vezesse le az ILU algoritmust. Írja át reziduumvektoros alakra is. Adjon nevezetes példákat az algoritmus speciális eseteiként.
b) Vázolja a részleges LU-felbontás előállításának algoritmusát. Adjon elégséges feltételt a felbontás létezésére és egyértelműségére.
c) * A felbontást előállító algoritmus helyességét megalapozó tétel bizonyítása.
17. Nemlineáris egyenletek megoldása I.
a) Ismertesse a nemlineáris egyenletek megoldásának feladatát. Mutassa be a megoldás létezését biztosító állításokat: írja fel a Bolzano-tételt (bizonyítás nélkül), és
segítségével igazolja a Brouwer-féle fixpont-tételt.
b) Kontrakció fogalma [a; b] intervallumon és a Banach-féle fixponttétel (bizonyítás
nélkül). Igazolja az elégséges feltételt a kontrakcióra.
c) * Ismertesse a konvergenciarend fogalmát és elemi tulajdonságait. Igazolja a fixpontiterációk magasabb rendű konvergenciájáról szóló tételt.
18. Nemlineáris egyenletek megoldása II.
a) Vázolja az intervallumfelezés algoritmusát és mutasson hozzá hibabecslést. Ismertesse a húrmódszer alapötletét, szemléltesse működését és vezesse le az algoritmusát.
b) Ismertesse a Newton-módszer alapötletét, szemléltesse működését és vezesse le a
képletét. Mutassa be a többváltozós esetet is. Milyen tételt ismer a módszer monoton konvergenciájáról (bizonyítás nélkül)?
c) * Igazolja a Newton-módszer monoton konvergenciájáról szóló állítást.
19. Nemlineáris egyenletek megoldása III.
a) Ismertesse a Newton-módszer alapötletét, szemléltesse működését és vezesse le a
képletét. Mutassa be a többváltozós esetet is. Milyen tételt ismer a módszer lokális
konvergenciájáról (bizonyítás nélkül)?
b) * Igazolja a Newton-módszer lokális konvergenciájáról szóló állítást.
c) Ismertesse a szelőmódszer alapötletét, szemléltesse működését és vezesse le az algoritmusát. Adjon konvergenciatételt (bizonyítás nélkül). Vesse össze az eredményeket
a Newton-módszerről tanultakkal.

\fi